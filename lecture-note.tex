\documentclass[12pt,a4paper]{book}
\usepackage[utf8]{inputenc}
\usepackage[T1]{fontenc}
\usepackage{amsmath,amsfonts,amssymb}
\usepackage{geometry}
\usepackage{hyperref}
\usepackage{tocloft}
\usepackage{titlesec}

% Page geometry
\geometry{margin=1in}

% Hyperref setup
\hypersetup{
    colorlinks=true,
    linkcolor=blue,
    filecolor=magenta,      
    urlcolor=cyan,
    pdftitle={Quantum Information Science},
    pdfauthor={Ruang Kuantum},
}

% Chapter and section formatting
\titleformat{\chapter}[display]
{\normalfont\huge\bfseries}{\chaptertitlename\ \thechapter}{20pt}{\Huge}
\titleformat{\section}
{\normalfont\Large\bfseries}{\thesection}{1em}{}
\titleformat{\subsection}
{\normalfont\large\bfseries}{\thesubsection}{1em}{}
\titleformat{\subsubsection}
{\normalfont\normalsize\bfseries}{\thesubsubsection}{1em}{}

% Custom table of contents formatting
\renewcommand{\cftchapfont}{\bfseries}
\renewcommand{\cftsecfont}{\normalfont}
\renewcommand{\cftsubsecfont}{\normalfont}
\renewcommand{\cftsubsubsecfont}{\normalfont}

\title{{\Huge Quantum Information Science}\\
{\Large Comprehensive Lecture Note}}
\author{Ruang Kuantum}
\date{\today}

\begin{document}

\frontmatter
\maketitle

\tableofcontents

\mainmatter
% --- CHAPTER 1: Foundations of Classical and Quantum Information ---
\chapter{From Classical to Quantum Information}

\section{Classical Information Fundamentals}
\subsection{Classical Bits and Logic Gates}
\subsection{Shannon's Entropy and Review}

\section{Introducing the Qubit (The Quantum Bit)}
\subsection{The Qubit: An Information Unit}
\subsection{Superposition (The State Space)}
\subsection{Entanglement (The Correlation)}

\section{The Rules of Quantum Computation}
\subsection{Quantum Measurement and Collapse}
\subsection{No-Cloning Theorem: Why It Matters}

% --- CHAPTER 2: Essential Linear Algebra Toolkit ---
\chapter{Essential Linear Algebra Toolkit}

\section{Complex Numbers Primer}
\subsection{Arithmetic and Polar Form}
\subsection{Visualizing Complex Numbers}

\section{Vectors and Quantum States}
\subsection{Vector Spaces and Basis Vectors}
\subsection{Ket Notation ($|\psi\rangle$)}
\subsection{Inner Products and Normalization}

\section{Matrices and Quantum Gates}
\subsection{Matrix Multiplication Review}
\subsection{Unitary Matrices}

\section{Composite Systems}
\subsection{Tensor Products ($\otimes$)}
\subsection{Combining $n$ Qubits}

% --- CHAPTER 3: Quantum Circuits and Single-Qubit Gates ---
\chapter{Quantum Circuits and Single-Qubit Gates}

\section{The Quantum Circuit Model}
\subsection{Circuit Notation and Wire Diagram}
\subsection{Reversibility and Unitarity}

\section{Key Single-Qubit Gates}
\subsection{The Pauli Gates ($X, Y, Z$)}
\subsection{The Hadamard Gate ($H$)}
\subsection{Phase and $T$ Gates (for Universal Sets)}

\section{The Bloch Sphere Representation}
\subsection{Visualizing a Single Qubit State}
\subsection{Rotation Gates ($R_x, R_y, R_z$)}

% --- CHAPTER 4: Multi-Qubit Gates and Universality ---
\chapter{Multi-Qubit Gates and Universality}

\section{Entangling Gates}
\subsection{CNOT (Controlled-NOT) Gate}
\subsection{Controlled Phase and SWAP Gates}

\section{Universal Gate Sets}
\subsection{What Does Universality Mean?}
\subsection{Clifford + T Set}

\section{Simple Quantum Algorithms (Examples)}
\subsection{Deutsch-Jozsa Algorithm}
\subsection{Quantum Parallelism in Practice}

% --- CHAPTER 5: Quantum Algorithms I: Searching and Factoring ---
\chapter{Quantum Algorithms I: Searching and Factoring}

\section{Mathematical Tool: Quantum Fourier Transform (QFT)}
\subsection{Definition and Purpose}
\subsection{Circuit Implementation Overview}

\section{Phase Estimation Algorithm}
\subsection{The Core Idea}
\subsection{Applications (Period Finding)}

\section{Shor's Factoring Algorithm}
\subsection{Problem of Factoring}
\subsection{The Algorithm's Structure}

\section{Grover's Search Algorithm}
\subsection{Unstructured Search Problem}
\subsection{Amplitude Amplification}
\subsection{Comparison: Speedup vs Classical}

% --- CHAPTER 6: Quantum Communication and Information Theory ---
\chapter{Quantum Communication and Information Theory}

\section{Quantum Information Compression}
\subsection{Von Neumann Entropy (Simple Definition)}
\subsection{Quantum Data Compression (Schumacher)}

\section{Communication Protocols}
\subsection{Quantum Teleportation}
\subsection{Superdense Coding}

\section{Security Through QKD}
\subsection{BB84 Protocol (Discrete Variable QKD)}
\subsection{E91 (Entanglement-Based QKD)}

% --- CHAPTER 7: NISQ Algorithms and Hybrid Computation ---
\chapter{Quantum Algorithms II: NISQ Algorithms and Hybrid Computation}

\section{The NISQ Era}
\subsection{Hardware Limitations and Noise Models}
\subsection{The Need for Hybrid Algorithms}

\section{Variational Quantum Eigensolver (VQE)}
\subsection{Algorithm Framework}
\subsection{Classical Optimization Loop}

\section{Quantum Approximate Optimization Algorithm (QAOA)}
\subsection{Optimization Problems (MaxCut, etc.)}
\subsection{Circuit Structure and Parameterization}

\section{Quantum Machine Learning (QML)}
\subsection{Hybrid Training Approaches}

% --- CHAPTER 8: Quantum Sensing and Metrology ---
\chapter{Quantum Sensing and Metrology}

\section{Limits of Measurement}
\subsection{Classical Noise and the Shot Noise Limit}
\subsection{Quantum Enhanced Sensing}
\subsection{Heisenberg Limit}

\section{Key Techniques}
\subsection{NOON States (for Phase Estimation)}
\subsection{Squeezed States (Intuitive)}

\section{Specific Sensing Applications}
\subsection{Quantum Magnetometry (NV-Centers)}
\subsection{Clock Synchronization}

% --- CHAPTER 9: Open Systems and Decoherence ---
\chapter{Open Systems and Decoherence}

\section{System-Environment Interaction}
\subsection{What is Decoherence?}
\subsection{Amplitude and Phase Damping Models}

\section{The Density Matrix}
\subsection{Representing Mixed States}
\subsection{Pure vs. Mixed States}

\section{Introduction to Quantum Control}
\subsection{Quantum Zeno Effect (Concept)}
\subsection{Dynamical Decoupling (Concept)}

% --- CHAPTER 10: Quantum Error Correction (QEC) Fundamentals ---
\chapter{Quantum Error Correction (QEC) Fundamentals}

\section{Why QEC is Necessary}
\subsection{The Fragility of Quantum Information}
\subsection{Syndrome Measurement Concept}

\section{Basic Quantum Codes}
\subsection{The Three-Qubit Code (Bit-Flip)}
\subsection{Shor's Nine-Qubit Code (Overview)}

\section{Introduction to Stabilizer Codes}
\subsection{Stabilizer Formalism (Simplified)}
\subsection{Logical Qubits and Encoding}

% --- CHAPTER 11: Quantum Hardware Platforms ---
\chapter{Quantum Hardware Platforms}

\section{Key Hardware Requirements}
\subsection{Qubit Quality (Fidelity, Coherence Time)}
\subsection{Scaling and Connectivity}

\section{Major Qubit Technologies}
\subsection{Superconducting Qubits (Transmons)}
\subsection{Trapped Ion Systems}
\subsection{Photonic Quantum Computing}

\section{Control and Readout}
\subsection{Pulse Generation Overview}
\subsection{Measurement Process}

% --- CHAPTER 12: Quantum Networking and Future ---
\chapter{Quantum Networking and Future}

\section{Quantum Networks vs Internet}
\subsection{Goals and Challenges}
\subsection{Quantum Repeaters (Concept)}

\section{Distributed Quantum Services}
\subsection{Quantum Clock Synchronization}
\subsection{Secure Multiparty Computation}

\section{The Road to Fault Tolerance}
\subsection{Threshold Theorems (Intuitive)}
\subsection{The Quantum Ecosystem}
\end{document}